\documentclass[14pt]{exam}

\usepackage{amsmath}
\usepackage{amssymb}
\usepackage{xcolor}
\usepackage{diagbox}
\usepackage{float}


\title{Probability and Statistics. Weeks 13}
\date{}

\def\Var{{\textrm{Var}}\,}
\def\E{{\textrm{E}}\,}
\def\Exp{{\textrm{Exp}}\,}

\begin{document}
	\maketitle
	
	\begin{questions}
		\question
		(Walpole 9.4)
		The height of a random sample of 50 college students showed a mean of $174.5$ centimeters and a standard deviation of $6.9$ centimeters.
		
		\begin{parts}
			\part Construct a $98\%$ confidence interval for the mean height of all college students.
			\part What can we assert with $98\%$ confidence about the possible size of our error if we estimate the mean height of all students to be $174.5$ centimeters?
		\end{parts}
		
		\question
		(Walpole 9.12)
		A random sample of $10$ chocolate energy bars of a certain brand has, on average, $230$ calories per bar,
		with a standard deviation of $15$ calories. Construct a $99\%$ confidence interval for the true mean calorie content of this brand of energy bar. Assume that the distribution of the calorie content is approximately normal.
		
		\question
		(Walpole 9.14)
		The following measurements were recorded for the drying time, in hours, of a certain brand of latex paint:
		$$
		\begin{tabular}{ccccc}
		3.4 & 2.5 & 4.8 & 2.9 & 3.6\\
		2.8 & 3.3 & 5.6 & 3.7 & 2.8\\
		4.4 & 4.0 & 5.2 & 3.0 & 4.8\\
		
		\end{tabular}
		$$
		Assuming that the measurements represent a random sample from a normal population, find a $95\%$ prediction interval for the drying time for the next trial of the paint.
		
		\question
		(Walpole 9.19)
		A random sample of $25$ tablets of buffered aspirin contains, on average, $325.05$ mg of aspirin per tablet, with a standard deviation of $0.5$ mg. Find the $95\%$ tolerance limits that will contain $90\%$ of the tablet contents for this brand of buffered aspirin. Assume that the aspirin content is normally distributed.
		
		\question
		(Walpole 9.36)
		Two kinds of thread are being compared for strength. Fifty pieces of each type of thread are tested under similar conditions. Brand A has an average tensile strength of $78.3$ kilograms with a standard deviation of $5.6$ kilograms, while brand B has an average tensile strength of $87.2$ kilograms with a standard deviation of $6.3$ kilograms. Construct a $95\%$ confidence interval for the difference of the population means.
		
		\question
		(Walpole 9.40)
		
		In a study conducted at Virginia Tech on the development of ectomycorrhizal, a symbiotic relationship between the roots of trees and a fungus, in which minerals are transferred from the fungus to the trees and sugars from the trees to the fungus, $20$ northern red oak seedlings exposed to the fungus Pisolithus tinctorus were grown in a greenhouse. All seedlings were planted in the same type of soil and received the same amount of sunshine and water. Half received no nitrogen at planting time, to serve as a control, and the other half received $368$ ppm of nitrogen in the form NaNO3. The stem weights, in grams, at the end of $140$ days were recorded as follows:
		
		$$
			\begin{tabular}{cc}
				No nitrogen & Nitrogen\\ \hline
				0.32 & 0.26\\
				0.53 & 0.43\\
				0.28 & 0.47\\
				0.37 & 0.49\\
				0.47 & 0.52\\
				0.43 & 0.75\\
				0.36 & 0.79\\
				0.42 & 0.86\\
				0.38 & 0.62\\
				0.43 & 0.46
			\end{tabular}
		$$
		
		Construct a $95\%$ confidence interval for the difference in the mean stem weight between seedlings that re ceive no nitrogen and those that receive $368$ ppm of nitrogen. Assume the populations to be normally distributed with equal variances.
		
		\question
		(Walpole 9.42)
		An experiment reported in Popular science compared fuel economies for two types of similarly equipped diesel mini-trucks. Let us suppose that $12$ Volkswagen and 10 Toyota trucks were tested in 90-kilometer-per-hour steady-paced trials. If the 12 Volkswagen trucks averaged $16$ kilometers per liter with a standard deviation of $1$ kilometer per liter and the 10 Toyota trucks averaged $11$ kilometer per liter with a standard deviation of $0.8$ kilometers per liter, construct a $90\%$ confidence interval for the difference between the average kilometers per liter for these two mini-trucks. Assume that distance per liter for the truck models are approximately normally distributed with equal variances.
	\end{questions}


	\textit{To be continued...}
	
\end{document}